\documentclass[11pt]{article}
\usepackage{fullpage,amsmath}
\usepackage[pdftex,colorlinks=true,plainpages=false]{hyperref}
\begin{document}
\begin{center}
%{\Large \em DRAFT}\\
{\Large COMPUTER SCIENCE 50100}\\
{\Large Computing for Science and Engineering} \\
{\large FALL 2015} \\
ASSIGNMENT \# 2 (35 points) \\
{\small September 9}
%ASSIGNMENT \# 2 Solutions (35 points) \\
%{\small October 6}
\end{center}

\paragraph{Due Wednesday, September 30 at 10:30 am}

This assignment covers Section 2.3--2.6 of the class notes.

\begin{enumerate}

\item (8 points)  % sec 2.3
 The problem is to write a module {\tt funcf.py}
 implementing two versions of a function {\tt o}
 that takes as it arguments two
 functions {\tt f} and {\tt g}, each of which maps a {\tt float}
 to a {\tt float}.
 The function {\tt o} is to return the composition {\tt f}$\circ${\tt g}.
 \begin{enumerate}
 \item (5 points)
 The first implementation, call it {\tt o1},
 should make no use of the {\tt lambda} operator.
 \item (3 points)
   The second implementation, call it {\tt o2},
   should make no use the {\tt def} command---none at all.
 \end{enumerate}
 Here is a template with a unit test:
 \input{_funcf_py}


\item (10 points)  sec 2.5  \\ % X2  before 2014
 Write a script that,
 from the current working directory and every nested subdirectory,
 removes every file whose name ends with ``\verb|~|'',
 but {\em only if} there is another file in the same directory
 with the same name except for the tilde.
Possibly useful functions: {\tt os.getcwd}, {\tt os.path.join},
 {\tt os.remove}, {\tt os.walk}.
 Here is a testing program:
 \input{cleanTest_py}

\item (17 points)  sec 2.6 \\
 Construct a Python module {\tt randNo.py} that defines a {\tt RandNo} class.
 Given a file of probabilities $p_i$, $i = 1, 2,\ldots, n$,
 construct a function that generates $i$ with probability $p_i$.
 \begin{enumerate}
 \item The constructor is given a list of probabilities
  $p_i$, $i = 0, 2,\ldots, n-1$, where $p_i$ is the probability that
  an instance of a {\tt RandNo} has the value $i$
  and creates an independent ``data structure''
  that expedites instantiation and addition.
  If the probabilities do not add up to 1 give or take $10^-7$,
  raise a {\tt ValueError} with the format string
  {\tt 'Probabilities sum to \%.7f.'}.
 \item Define a call method that effects instantiation.
  The call should take only $\mathcal{O}(\log n)$ computing time
  where $n-1$ is the maximum value.
 \item Define a {\tt str} method for the {\tt str} function,
  as illustrated below.
 \item Define a {\tt repr} method identical to the {\tt str} method.
 \item Define a method that enables two {\tt RandNo}s to be added
  using the operator {\tt +}.
  The result should have the correct probability distribution.
 \end{enumerate}
Possibly useful functions:  {\tt random.random}.
 Here is a template with a unit test:
 \input{_randNo_py}

\end{enumerate}

\paragraph{Send your solutions electronically}
You should submit one PDF file containing your source code for problems~1--3.
Additionally,
send your source code for each problem as separate files.

\end{document}
